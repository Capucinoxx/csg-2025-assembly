\subsection*{Pointage}
\noindent
Soit l'ensemble des tests ayant été fournis de base pour le projet :
\begin{itemize}
  \item \texttt{manuals\_f1/} : dossier contenant les manuels 
    d'instructions avec le premier format pour les instructions.
  \item \texttt{manuals\_f2/} : dossier contenant les manuels 
    d'instructions avec le deuxième format pour les instructions.
\end{itemize}
\faExclamationCircle\ \ Un (1) seul fichier par format d'instructions
sera utilisé pour les tests. Puisqu'il y a deux formats, vous aurez
deux fichiers à soumettre. Si plusieurs (plus de deux) fichiers sont 
soumis, un fichier sera choisi aléatoirement pour chaque format.
\medskip

\noindent
\textbf{Notation .} \quad
Pour chacun des manuels (des dossiers \texttt{manuals\_f1} et 
\texttt{manuals\_f2}), un test sera effectué pour vérifier que le 
programme donne bien le résultat attendu (dessin dans la grille 
\texttt{MAP}). 

\noindent
Attribution des points pour chaque manuel :
\begin{itemize}
  \item \underline{Si c'est le bon résultat attendu :} \\
    L'équipe ayant fait l'implémentation utilisant le moins de cycle
    d'horloge aura un (1) point pour les manuels \texttt{m1\_x}, 
    deux (2) points pour les manuels \texttt{m2\_x} ainsi de suite. 
    L'équipe ayant fait l'implémentation la plus lente aura trois 
    quarts (0.75) de point de moins que le maximum 0.25 pour les 
    manuels \texttt{m1\_x}, 1.25pour les manuels \texttt{m2\_x}, ainsi 
    de suite. Les autres équipes ayant réussi auront un pointage 
    distribué de manière linéaire entre ces deux bornes.
  \item \underline{Sinon :} \\
    Un pointage variant entre zéro (0) et un dixième (0.1) de point 
    sera attribué selon l'avancement de la construction.
  \item \underline{Si le programme est hardcodé :} \\
    Un pointage de 0.25 sera attribué.
\end{itemize}

\noindent
Le pointage final sera composé de la sommation des points de chaque 
manuel.
