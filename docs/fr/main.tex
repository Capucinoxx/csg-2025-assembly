\documentclass[10pt, letterpaper]{article}
\usepackage{../common/macro}
\usepackage{../common/styles}

\usepackage[french]{babel}

\begin{document}
  \thispagestyle{empty} 
\begin{center}

  \vfill

  \vspace{5cm}

  {\huge Tester par soi-même apporte plus de réponses que mille 
  questions théoriques.}

  \vfill

  {\Large \textcolor{red!50!black}{\textbf{Aucune question permise}}}
\end{center}

\clearpage

  
  \begin{abstract}
    L'innovation en programmation repose sur la capacité à transcender
    les limitations des langages existants. Ces derniers, bien que 
    puissants en terme d'expressivité, peuvent restreindre la 
    créativité des développeur.euse.s en imposant des structures
    rigides. Cette note propose la création d'un langage assembleur
    basé sur l'arithmétique de Presburger, une théorie mathématique
    offrant un cadre décidable pour la manipulation des entiers.
  \end{abstract}

  \subsection*{Architecture du processeur}
\noindent
\textbf{Cycle du pipeline .} \quad
Chaque instruction suit un cycle de traitement en un maximum de 
\underline{cinq étapes}, selon le schéma suivant :
\begin{enumerate}
  \item \texttt{IF (Instruction Fetch)} - Chargement de l'instruction
    depuis la mémoire.
  \item \texttt{ID (Instruction Decode)} - Décodage de l'instruction
    pour identifier les opérandes et registres.
  \item \texttt{EX (Execution)} - Exécution des calculs arithmétiques,
    logiques ou autres opérations.
  \item \texttt{MEM (Memory Access)} - Accès à la mémoire pour lire ou
    écrire des données.
  \item \texttt{WB (Write Back)} - Écriture des résultats dans les
    registres.
\end{enumerate}
\medskip

\noindent
\textbf{Registres .} \quad
Le processeur dispose de $5$ registres $R_0, R_1, R_2, R_3, R_4$ non
signés de quatre (4) octets ainsi que certains registres spéciaux :
\begin{itemize}
  \item \texttt{pc (Program Counter)} - Registre pointant vers 
    l'instruction en cours d'exécution.
  \item \texttt{sp (Stack Pointer)} - Registre pointant vers le sommet
    de la pile.
  \item \texttt{lr (Link Register)} - Registre de sauvegarde de 
    l'adresse de retour.
\end{itemize}
\medskip

\noindent
\textbf{Pile .} \quad
Le processeur dispose d'une pile $P$ bornée à $32$ éléments de quatre
(4) octets non signés chacun. 

  \subsection*{Modèle de mémoire}
\noindent
La mémoire visible par un programme assembleur est segmentée en deux 
parties distinctes, soit la mémoire en lecture seule (ROM) et la 
mémoire en écriture seule (RAM). La mémoire est adressée par des 
entiers non signés de quatre (4) octets. Soit les mots-clés suivants :
\begin{itemize}
  \item \texttt{MEM (mémoire)} - Tableau d'octets représentant la 
    mémoire en lecture seule (ROM).
  \item \texttt{MAP (mappage)} - Tableau d'octets représentant la 
    mémoire en écriture seule (RAM).
\end{itemize}


  \subsection*{Instruction Set}
\begin{table}[h]
\centering
\label{tab:instructions}
\begin{tabular}{llccccc}
  \toprule
  \emph{Instruction} & \emph{Description} & \emph{IF} & \emph{ID} 
                     & \emph{EX} & \emph{MEM} & \emph{WB} \\

  \bottomrule
  \multicolumn{7}{l}{\textbf{Stack}} \\
  \toprule
  \texttt{PUSH Rx} & $P[sp] \gets R_x;\quad sp \gets sp + 1$ 
                   & \yes & \yes & & \yes & \\
  \texttt{POP Rx}  & $R_x \gets P[sp];\quad sp \gets sp - 1$ 
                   & \yes & & & \yes & \yes \\
  \texttt{TOP Rx}  & $R_x \gets P[sp]$ 
                   & \yes & & & \yes & \yes \\
  \texttt{SWAP}    & $P[sp] \leftrightarrow P[sp-1]$ 
                   & \yes & & & & \yes \\

  \bottomrule
  \multicolumn{7}{l}{\textbf{Arithmetic}} \\
  \toprule
  \texttt{ADD Rx, Ry, Rz} & $R_x \gets R_y + R_z$ 
                          & \yes & \yes & \yes & & \yes \\
  \texttt{MOV Rx, Ry}     & $R_x \gets R_y$
                          & \yes & \yes & & & \yes \\
  \texttt{INC Rx}         & $R_x \gets Rx + 1$
                          & \yes & \yes & \yes & & \yes \\
  \texttt{ZER Rx}         & $R_x \gets 0$ 
                          & \yes & & & & \yes \\
  \bottomrule
  \multicolumn{7}{l}{\textbf{Memory}} \\
  \toprule
  \texttt{LDR Rx, Ry, Rz} 
    & $R_x \gets MEM[R_y : R_y + min(R_z, 4)]$ 
    & \yes & \yes & \yes & \yes & \yes \\
  \texttt{STR Rx, Ry} 
    & $MAP[R_x : R_x + 4] \gets MAP[R_x : R_x + 4]\ |\ R_y $ 
    & \yes & \yes & \yes & \yes & \\

  \bottomrule
  \multicolumn{7}{l}{\textbf{Control}} \\
  \toprule
  \texttt{JLE Rx, Ry, label} 
    & $pc \leftarrow \text{label} \quad \text{if }\ \ x \leq y$
    & \yes & \yes & \yes & & \yes \\
  \texttt{CALL label} 
    & $lr \gets pc;\quad pc \gets$ \texttt{label} 
    & \yes & & & & \yes \\
  \texttt{RET}  & $pc \gets lr$ 
                & \yes & \yes & & & \yes \\
  \texttt{NOP}  & \emph{Does nothing} 
                & \yes & & & & \\
  \bottomrule
\end{tabular}
\end{table}


  
  \clearpage
  \noindent
Ainsi, l'objectif est de démontrer que le langage assembleur minimal 
présenté est suffisamment puissant pour permettre l'expression d'un 
calcul permettant l'assemblage de blocs deux dimensions à l'aide d'un 
manuel d'instructions.
\medskip

\noindent
Il existe deux formats de manuel d'instructions pouvant être 
interprété par le langage $L'$ :
\begin{enumerate}
  \item \emph{premier format}
    \begin{itemize}
      \item \texttt{1er octet} : largeur (4 bits) et hauteur (4 bits) 
        de la pièce.
      \item \texttt{2e octet} : décalage horizontal (4 bits) et 
        vertical (4 bits) de la pièce.
      \item \texttt{3e octet} : couleur de la pièce.
    \end{itemize}
    Soit par exemple $\texttt{MEM} {:=} [0x11, 0x22, 0x07, 0x21, 0x10, 
    0x06]$ pouvant être traduit par :
    \begin{enumerate}
      \item[1.] Placer une pièce $1 \times 1$ avec un déplacement de 
          $(2, 2)$ et de couleur $0x07$.
      \item [2.] Placer une pièce $2 \times 1$ avec un déplacement de 
          $(1, 0)$ et de couleur $0x06$.
    \end{enumerate}
  \item \emph{deuxième format}
    \begin{itemize}
      \item \texttt{1er octet} : largeur de la pièce.
      \item \texttt{2e octet} : hauteur de la pièce.
      \item \texttt{3e octet} : décalage horizontal de la pièce.
      \item \texttt{4e octet} : décalage vertical de la pièce.
      \item \texttt{5e octet} : couleur de la pièce.
    \end{itemize}
    Soit par exemple $\texttt{MEM} {:=} [0x01, 0x01, 0x02, 0x02, 0x07, 
    0x02, 0x01, 0x01, 0x00, 0x06]$ pouvant être traduit par :
    \begin{enumerate}
      \item[1.] Placer une pièce $1 \times 1$ avec un déplacement de 
          $(2, 2)$ et de couleur $0x07$.
      \item [2.] Placer une pièce $2 \times 1$ avec un déplacement de 
          $(1, 0)$ et de couleur $0x06$.
    \end{enumerate}
\end{enumerate}
Le programme assembleur doit être capable de lire un manuel 
d'instructions (se trouvant dans MEM) dans l'un des deux formats et 
d'assembler les blocs de manière à former une construction en deux 
dimensions.
\medskip

\noindent
Le positionnement suit une logique cumulative : initialement à $(0, 
0)$, soit le coin supérieur gauche, chaque pièce est placée 
relativement à la précédente. Son ancrage est fixé à son coin 
supérieur gauche. Pour que cela reste décidable, la taille de la 
grille servant à y placer les pièces (\texttt{MAP}) est fixée à $8 
\times 8$. Lorsqu'un déplacement occasionne un dépassement de la 
grille, le programme doit repositionner son ancrage à la position
$0$ de cette dimension. C'est-à-dire que si une pièce subit un 
déplacement de $(4, 3)$ et que l'on se trouve à la position $(5, 6)$,
la pièce sera placée à la position $((5 + 4) \mod 8, (6 + 3) \mod 8) 
= (1, 1)$. Cette même logique est appliquée lorsqu'une partie de la 
pièce dépasse la grille. Le point d'ancrage finale après la mise en 
place d'une pièce est donnée par la formule suivante : $((p_{ix}
 + w + dx) \mod 8, (p_{iy} + h + dy - 1) \mod 8)$ où $p_{ix}$ et 
 $p_{iy}$ sont les coordonnées du coin supérieur gauche du point 
d'ancrage initial, $w$ et $h$ sont respectivement la largeur et la
hauteur de la pièce et $dx$ et $dy$ sont les déplacements horizontal
et vertical de la pièce.
\medskip

\noindent
L'illustration ci-dessous montre l'évolution de la grille après le 
placement des pièces pour la mémoire \\
$\texttt{MEM} {:=} [0x11, 0x22, 0x07, 0x21, 0x00, 0x06]$ (premier 
format):
\medskip

\begin{minipage}{0.2\textwidth}
{\scriptsize Initialement : }\\
\begin{tikzpicture}[scale=0.275]
  \foreach \x in {0, 1, ..., 7} {
    \foreach \y in {0, 1, ..., 7} {
      \draw[fill=white] (\x, \y) rectangle (\x + 1, \y + 1);
    }
  }

  \draw[fill=gray] (0, 8) circle (0.15);
\end{tikzpicture}
\end{minipage}
\begin{minipage}{0.2\textwidth}
{\scriptsize Après la première pièce :} \\
\begin{tikzpicture}[scale=0.275]
  \foreach \x in {0, 1, ..., 7} {
    \foreach \y in {0, 1, ..., 7} {
      \draw[fill=white] (\x, \y) rectangle (\x + 1, \y + 1);
    }
  }

  \draw[fill=gray] (3, 6) circle (0.15);
  \node[font=\tiny] at (2.5, 5.5) {7};
\end{tikzpicture}
\end{minipage}
\begin{minipage}{0.33\textwidth}
  {\scriptsize Après la deuxième pièce :} \\
\begin{tikzpicture}[scale=0.275]
  \foreach \x in {0, 1, ..., 7} {
    \foreach \y in {0, 1, ..., 7} {
      \draw[fill=white] (\x, \y) rectangle (\x + 1, \y + 1);
    }
  }

  \draw[fill=gray] (5, 6) circle (0.15);
  \node[font=\tiny] at (2.5, 5.5) {7};
  \node[font=\tiny] at (3.5, 5.5) {6};
  \node[font=\tiny] at (4.5, 5.5) {6};
\end{tikzpicture}
\end{minipage}
\medskip

\noindent
L'objectif est donc dans un premier temps de traduire un manuel 
d'instructions pour en faire une construction dans la grille 
\texttt{MAP}. Dans un second temps, nous conjecturons qu'il existe
plusieurs agencements d'instructions permettant de réaliser un 
programme assembleur minimal $L'$ capable de réaliser cette tâche. Il 
faudrait donc optimiser le programme pour que ce dernier utilise le 
moins de cycle possible avant de terminer la construction.


  \clearpage 
  \subsection*{Outil de déverminage}
\noindent
Un outil rudimentaire a été mis sur place pour aider à la conception
de programmes en langage assembleur. Cet outil permet de visualiser
l'état du programme à chaque étape de son exécution. Ce programme 
nommé \emph{asm} s'utilise comme suit :
\begin{verbatim}
  ./asm [-p <path>] [-m <path>] [-b breakpoints]
\end{verbatim}
Où 
\begin{itemize}
  \item \texttt{-p <path>} : spécifie le chemin vers le programme 
    assembleur à exécuter.
  \item \texttt{-m <path>} : spécifie le chemin vers le manuel 
    d'instructions.
  \item \texttt{-b breakpoints} : spécifie les points d'arrêt séparés 
    par des virgules. On doit écrire les numéros de lignes faisant
    référence au fichier assembleur.
\end{itemize}
Exemple d'utilisation :
\verb|./asm -p program.asm -m manual.txt -b 3,6,42|
\medskip

\noindent
\textbf{Visuel du programme. } \quad
Le programme de déverminage est composé de six zones distinctes ainsi
que six commandes pour interagir avec le programme. Voici sous forme 
de tableaux les différentes zones et commandes disponibles :

\begin{center}
\begin{tabular}{cl}
  \toprule
  \emph{Commande} & \emph{Utilité} \\
  \midrule
  \texttt{q} & Quitter le programme. \\
  \texttt{i} & Changer la zone sélectionnée. \\
  \texttt{w} & Défiler vers le haut la zone sélectionnée. \\
  \texttt{x} & Défiler vers le bas la zone sélectionnée. \\
  \texttt{s} & Exécuter une instruction. \\
  \texttt{c} & Continuer l'exécution jusqu'au prochain point d'arrêt. \\
  \bottomrule
\end{tabular} 
\medskip

Liste des commandes pour interagir avec le dévermineur.
\end{center}
\medskip

\begin{center}
\begin{tabular}{ll}
  \toprule
  \emph{Zone} & \emph{Description} \\
  \midrule
  \texttt{Instructions} & Liste des instructions du programme. \\
  \texttt{Processor} & Information sur la pile et les registres. \\
  \texttt{Memory} & Contenu de la mémoire ROM (manuel d'instructions). \\
  \texttt{Map} & Contenu de la mémoire RAM (grille de construction). \\
  \texttt{Pipeline} & État du pipeline d'exécution. \\
  \bottomrule
\end{tabular}
\medskip

Liste des zones affichées par le dévermineur.
\end{center}
\medskip

\noindent
Pour la zone \texttt{Map}, les couleurs seront représentées non pas 
par des valeurs hexadécimales, mais par leur équivalent en couleur.
\[
  0 \Rightarrow \text{Blanc, } 1 \Rightarrow \text{Bleu, }
  2 \Rightarrow \text{Vert, } 3 \Rightarrow \text{Cyan, }
  4 \Rightarrow \text{Rouge, } 5 \Rightarrow \text{Magenta, }
  6 \Rightarrow \text{Jaune, } 7 \Rightarrow \text{Noir}
\]
\medskip

\noindent
\textbf{Manuels d'instructions. } \quad
Différents manuels d'instructions sont fournis pour tester vos 
programmes. Voici comment ils sont structurés :
\begin{itemize}
  \item \emph{première ligne} : numéro de format du manuel (1 ou 2).
  \item \emph{lignes suivantes} : chaque ligne contient une 
    instruction sous la forme \\ 
    \texttt{<largeur> <hauteur> <décalage\_horizontal> 
    <décalage\_vertical> <couleur>}.
\end{itemize}
On peut prendre pour acquis que les instructions ont des valeurs 
valides et minimales. Par exemple, un décalage horizontal ne dépassera
jamais la largeur - 1 de la grille. 


  \clearpage
  \subsection*{Pointage}
\noindent
Soit l'ensemble des tests ayant été fournis de base pour le projet :
\begin{itemize}
  \item \texttt{manuals\_f1/} : dossier contenant les manuels 
    d'instructions avec le premier format pour les instructions.
  \item \texttt{manuals\_f2/} : dossier contenant les manuels 
    d'instructions avec le deuxième format pour les instructions.
\end{itemize}
\faExclamationCircle\ \ Un (1) seul fichier par format d'instructions
sera utilisé pour les tests. Puisqu'il y a deux formats, vous aurez
deux fichiers à soumettre. Si plusieurs (plus de deux) fichiers sont 
soumis, un fichier sera choisi aléatoirement pour chaque format.
\medskip

\noindent
\textbf{Notation .} \quad
Pour chacun des manuels (des dossiers \texttt{manuals\_f1} et 
\texttt{manuals\_f2}), un test sera effectué pour vérifier que le 
programme donne bien le résultat attendu (dessin dans la grille 
\texttt{MAP}). 

\noindent
Attribution des points pour chaque manuel :
\begin{itemize}
  \item \underline{Si c'est le bon résultat attendu :} \\
    L'équipe ayant fait l'implémentation utilisant le moins de cycle
    d'horloge aura un (1) point pour les manuels \texttt{m1\_x}, 
    deux (2) points pour les manuels \texttt{m2\_x} ainsi de suite. 
    L'équipe ayant fait l'implémentation la plus lente aura trois 
    quarts (0.75) de point de moins que le maximum 0.25 pour les 
    manuels \texttt{m1\_x}, 1.25pour les manuels \texttt{m2\_x}, ainsi 
    de suite. Les autres équipes ayant réussi auront un pointage 
    distribué de manière linéaire entre ces deux bornes.
  \item \underline{Sinon :} \\
    Un pointage variant entre zéro (0) et un dixième (0.1) de point 
    sera attribué selon l'avancement de la construction.
  \item \underline{Si le programme est hardcodé :} \\
    Un pointage de 0.25 sera attribué.
\end{itemize}

\noindent
Le pointage final sera composé de la sommation des points de chaque 
manuel.

\end{document}
