\subsection*{Architecture du processeur}
\noindent
\textbf{Cycle du pipeline .} \quad
Chaque instruction suit un cycle de traitement en un maximum de 
\underline{cinq étapes}, selon le schéma suivant :
\begin{enumerate}
  \item \texttt{IF (Instruction Fetch)} - Chargement de l'instruction
    depuis la mémoire.
  \item \texttt{ID (Instruction Decode)} - Décodage de l'instruction
    pour identifier les opérandes et registres.
  \item \texttt{EX (Execution)} - Exécution des calculs arithmétiques,
    logiques ou autres opérations.
  \item \texttt{MEM (Memory Access)} - Accès à la mémoire pour lire ou
    écrire des données.
  \item \texttt{WB (Write Back)} - Écriture des résultats dans les
    registres.
\end{enumerate}
\medskip

\noindent
\textbf{Registres .} \quad
Le processeur dispose de $5$ registres $R_0, R_1, R_2, R_3, R_4$ non
signés de quatre (4) octets ainsi que certains registres spéciaux :
\begin{itemize}
  \item \texttt{pc (Program Counter)} - Registre pointant vers 
    l'instruction en cours d'exécution.
  \item \texttt{sp (Stack Pointer)} - Registre pointant vers le sommet
    de la pile.
  \item \texttt{lr (Link Register)} - Registre de sauvegarde de 
    l'adresse de retour.
\end{itemize}
\medskip

\noindent
\textbf{Pile .} \quad
Le processeur dispose d'une pile $P$ bornée à $32$ éléments de quatre
(4) octets non signés chacun. 
