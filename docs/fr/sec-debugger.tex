\subsection*{Outil de déverminage}
\noindent
Un outil rudimentaire a été mis sur place pour aider à la conception
de programmes en langage assembleur. Cet outil permet de visualiser
l'état du programme à chaque étape de son exécution. Ce programme 
nommé \emph{asm} s'utilise comme suit :
\begin{verbatim}
  ./asm [-p <path>] [-m <path>] [-b breakpoints]
\end{verbatim}
Où 
\begin{itemize}
  \item \texttt{-p <path>} : spécifie le chemin vers le programme 
    assembleur à exécuter.
  \item \texttt{-m <path>} : spécifie le chemin vers le manuel 
    d'instructions.
  \item \texttt{-b breakpoints} : spécifie les points d'arrêt séparés 
    par des virgules. On doit écrire les numéros de lignes faisant
    référence au fichier assembleur.
\end{itemize}
Exemple d'utilisation :
\verb|./asm -p program.asm -m manual.txt -b 3,6,42|
\medskip

\noindent
\textbf{Visuel du programme. } \quad
Le programme de déverminage est composé de six zones distinctes ainsi
que six commandes pour interagir avec le programme. Voici sous forme 
de tableaux les différentes zones et commandes disponibles :

\begin{center}
\begin{tabular}{cl}
  \toprule
  \emph{Commande} & \emph{Utilité} \\
  \midrule
  \texttt{q} & Quitter le programme. \\
  \texttt{i} & Changer la zone sélectionnée. \\
  \texttt{w} & Défiler vers le haut la zone sélectionnée. \\
  \texttt{x} & Défiler vers le bas la zone sélectionnée. \\
  \texttt{s} & Exécuter une instruction. \\
  \texttt{c} & Continuer l'exécution jusqu'au prochain point d'arrêt. \\
  \bottomrule
\end{tabular} 
\medskip

Liste des commandes pour interagir avec le dévermineur.
\end{center}
\medskip

\begin{center}
\begin{tabular}{ll}
  \toprule
  \emph{Zone} & \emph{Description} \\
  \midrule
  \texttt{Instructions} & Liste des instructions du programme. \\
  \texttt{Processor} & Information sur la pile et les registres. \\
  \texttt{Memory} & Contenu de la mémoire ROM (manuel d'instructions). \\
  \texttt{Map} & Contenu de la mémoire RAM (grille de construction). \\
  \texttt{Pipeline} & État du pipeline d'exécution. \\
  \bottomrule
\end{tabular}
\medskip

Liste des zones affichées par le dévermineur.
\end{center}
\medskip

\noindent
Pour la zone \texttt{Map}, les couleurs seront représentées non pas 
par des valeurs hexadécimales, mais par leur équivalent en couleur.
\[
  0 \Rightarrow \text{Blanc, } 1 \Rightarrow \text{Bleu, }
  2 \Rightarrow \text{Vert, } 3 \Rightarrow \text{Cyan, }
  4 \Rightarrow \text{Rouge, } 5 \Rightarrow \text{Magenta, }
  6 \Rightarrow \text{Jaune, } 7 \Rightarrow \text{Noir}
\]
\medskip

\noindent
\textbf{Manuels d'instructions. } \quad
Différents manuels d'instructions sont fournis pour tester vos 
programmes. Voici comment ils sont structurés :
\begin{itemize}
  \item \emph{première ligne} : numéro de format du manuel (1 ou 2).
  \item \emph{lignes suivantes} : chaque ligne contient une 
    instruction sous la forme \\ 
    \texttt{<largeur> <hauteur> <décalage\_horizontal> 
    <décalage\_vertical> <couleur>}.
\end{itemize}
On peut prendre pour acquis que les instructions ont des valeurs 
valides et minimales. Par exemple, un décalage horizontal ne dépassera
jamais la largeur - 1 de la grille. 
