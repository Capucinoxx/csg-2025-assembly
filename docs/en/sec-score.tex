\subsection*{Ranking}
\noindent
Let the set of tests provided by default for the project be as 
follows:
\begin{itemize}
  \item \texttt{manuals\_f1/} : folder containing the instruction 
    manuals with the first format for the instructions.
  \item \texttt{manuals\_f2/} : folder containing the instruction 
    manuals with the second format for the instructions.
\end{itemize}
\faExclamationCircle\ \ Only one (1) file per instruction format
will be used for the tests. Since there are two formats, you will need 
to submit two files. If multiple (more than two) files are submitted, 
one file will be randomly chosen for each format.
\medskip

\noindent
\textbf{Scoring. } \quad
For each of the manuals (from the \texttt{manuals\_f1} and 
\texttt{manuals\_f2} folders), a test will be performed to verify that 
the program gives the expected result (drawing in the \texttt{MAP} 
grid).

\noindent
Point allocation for each manual:
\begin{itemize} 
  \item \underline{If it is the correct expected result:} \\ 
    The team that implemented the solution using the least number of 
    clock cycles will receive one (1) point for the \texttt{m1\_x} 
    manuals, two (2) points for the \texttt{m2\_x} manuals, and so on. 
    The team that implemented the slowest solution will receive 
    three-quarters (0.75) of a point less than the maximum, i.e., 
    0.25 for the \texttt{m1\_x} manuals, 1.25 for the \texttt{m2\_x} 
    manuals, and so on. Other teams that succeeded will have their 
    score distributed linearly between these two bounds. 
  \item \underline{Otherwise:} \\ 
    A score ranging from zero (0) to one-tenth (0.1) of a point will 
    be awarded based on the progress of the construction. 
  \item \underline{If the program is hardcoded:} \\ A score of 0.25 
    will be awarded. 
\end{itemize}

\noindent
The final score will be the sum of the points from each manual.
