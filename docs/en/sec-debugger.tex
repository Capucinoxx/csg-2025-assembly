\subsection*{Debugging Tool}
\noindent
A rudimentary tool has been put in place to assist in the design
of programs in assembly language. This tool allows you to visualize
the state of the program at each step of its execution. This program,
called \emph{asm}, is used as follows:
\begin{verbatim}
  ./asm [-p <path>] [-m <path>] [-b breakpoints]
\end{verbatim}
Where
\begin{itemize}
  \item \texttt{-p <path>} : specifies the path to the assembly 
    program to be executed.
  \item \texttt{-m <path>} : specifies the path to the instruction 
    manual.
  \item \texttt{-b breakpoints} : specifies the breakpoints separated 
    by commas. Line numbers referring to the assembly file should be 
    written.
\end{itemize}
Example of usage:
\verb|./asm -p program.asm -m manual.txt -b 3,6,42|
\medskip

\noindent
\textbf{Program Visualization. } \quad
The debugging program consists of six distinct areas as well 
as six commands to interact with the program. Below are tables listing
the different available areas and commands:

\begin{center}
\begin{tabular}{cl}
  \toprule
  \emph{Command} & \emph{Purpose} \\
  \midrule
  \texttt{q} & Quit the program. \\
  \texttt{i} & Change the selected area. \\
  \texttt{w} & Scroll up the selected area. \\
  \texttt{x} & Scroll down the selected area. \\
  \texttt{s} & Execute an instruction. \\
  \texttt{c} & Continue execution until the next breakpoint. \\
  \bottomrule
\end{tabular} 
\medskip

List of commands to interact with the debugger.
\end{center}
\medskip

\begin{center}
\begin{tabular}{ll}
  \toprule
  \emph{Area} & \emph{Description} \\
  \midrule
  \texttt{Instructions} & List of program instructions. \\
  \texttt{Processor} & Information about the stack and registers. \\
  \texttt{Memory} & Content of the ROM memory (instruction manual). \\
  \texttt{Map} & Content of the RAM memory (construction grid). \\
  \texttt{Pipeline} & Status of the execution pipeline. \\
  \bottomrule
\end{tabular}
\medskip

List of areas displayed by the debugger.
\end{center}
\medskip

\noindent
For the \texttt{Map} area, the colors will be represented not by 
hexadecimal values, but by their equivalent color.
\[
  0 \Rightarrow \text{White, } 1 \Rightarrow \text{Blue, }
  2 \Rightarrow \text{Green, } 3 \Rightarrow \text{Cyan, }
  4 \Rightarrow \text{Red, } 5 \Rightarrow \text{Magenta, }
  6 \Rightarrow \text{Yellow, } 7 \Rightarrow \text{Black}
\]
\medskip

\noindent
\textbf{Instruction Manuals. } \quad
Different instruction manuals are provided to test your 
programs. Here is how they are structured:
\begin{itemize}
  \item \emph{First line} : manual format number (1 or 2).
  \item \emph{Following lines} : each line contains an instruction in 
    the form of \\ \texttt{<width> <height> <horizontal\_offset> 
    <vertical\_offset> <color>}.
\end{itemize}
It can be assumed that the instructions have valid and minimum values. 
For example, a horizontal offset will never exceed the width - 1 of 
the grid.

