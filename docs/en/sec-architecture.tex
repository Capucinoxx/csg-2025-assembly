\subsection*{Processor Architecture}
\noindent
\textbf{Pipeline Cycle.} \quad
Each instruction follows a processing cycle of up to \underline{five 
steps}, according to the following scheme:
\begin{enumerate}
  \item \texttt{IF (Instruction Fetch)} - Loading the instruction 
    from memory.
  \item \texttt{ID (Instruction Decode)} - Decoding the instruction 
    to identify operands and registers.
  \item \texttt{EX (Execution)} - Performing arithmetic, logical, or 
    other operations.
  \item \texttt{MEM (Memory Access)} - Accessing memory to read or 
    write data.
  \item \texttt{WB (Write Back)} - Writing results into registers.
\end{enumerate}
\medskip

\noindent
\textbf{Registers.} \quad
The processor has $5$ unsigned four-byte ($4$-byte) registers: $R_0, 
R_1, R_2, R_3, R_4$, as well as several special registers:
\begin{itemize}
  \item \texttt{pc (Program Counter)} - Register pointing to the 
    currently executing instruction.
  \item \texttt{sp (Stack Pointer)} - Register pointing to the top of 
    the stack.
  \item \texttt{lr (Link Register)} - Register for storing the return 
    address.
\end{itemize}

\noindent
\textbf{Stack.} \quad
The processor has a $P$ stack bounded by $32$ elements of four (4) 
unsigned bytes each. 
